\documentclass[12pt, titlepage=true, toc=bib]{scrartcl}

\usepackage[utf8]{inputenc} 											%Deutsche Silbentrennung
\usepackage[T1]{fontenc}												%Trennung von Wörtern mit Umlauten
\usepackage[ngerman]{babel}												%Deutsche Sprachregeln

\usepackage[bitstream-charter]{mathdesign}								%Schriftart
\usepackage[activate={true,nocompatibility},final,tracking=true,kerning=true,spacing=true,factor=1100,stretch=10,shrink=10]{microtype} 													
\addtokomafont{disposition}{\normalfont\bfseries}						%Selbe Schriftart für Überschriften
\usepackage[onehalfspacing]{setspace}									%Zeilenabstand

\usepackage[backend=biber, style=authoryear, doi=false, isbn=false]{biblatex}		%Literaturverwaltung
\usepackage[babel,german=quotes]{csquotes}								%Formatierung für Anführungszeichen
\addbibresource{Rechtsextremismus_und_Gender.bib}
\renewcommand{\labelnamepunct}{\addcolon\addspace}						%Formatiert die Zitationsklammern
\renewcommand{\postnotedelim}{\addcolon\addspace}
\DeclareFieldFormat{postnote}{#1}
\DeclareFieldFormat{multipostnote}{#1}
\AtBeginBibliography{%													%Macht den Namen im Literaturverzeichnis Fett
  \renewcommand*\mkbibnamefirst[1]{\bfseries{#1}}
  \renewcommand*\mkbibnamelast[1]{\bfseries{#1}}
  \renewcommand*\mkbibnameprefix[1]{\bfseries{#1}}
  \renewcommand*\mkbibnameaffix[1]{\bfseries{#1}}
}
\setlength{\bibitemsep}{0.5\baselineskip plus 0.5\baselineskip}			%Leerzeile zwischen den Einträgen im Literaturverzeichnis
\renewcommand{\multinamedelim}[0]{/}									%Schrägstrich bei zwei Autor*innen
\renewcommand{\finalnamedelim}[0]{/}


%\usepackage[left=2.5cm, right=3.5cm,]{geometry}							%Seitenränder

\begin{document}

\titlehead{Freie Universität Berlin\\
			FB Politik- und Sozialwissenschaften\\
			Otto-Suhr-Institut\\
			Sommersemester 2016\\
			{[}GEND{]} Rechtsextremismus und Gender: Geschichte und Gegenwart (15095)\\
			Carmen Altmeyer, Till Herold}
\author{Astrid Ursprung\and Ole Fechner}
\title{Catchy Title}
\subtitle{Mega wissenschaftlicher Untertitel oder Fragestellung}
\date{\normalsize{Berlin, den \today}}

\publishers{\normalsize{Goltzstr. 16\hfill Gerichtstr. 13\\
							 10781 Berlin\hfill 13347 Berlin\\
							 ursprung@posteo.de\hfill ole.fechner@fu-berlin.de\\
							 Matrikelnummer: 4768210\hfill Matrikelnummer: 4757766\\
							 BA Politikwissenschaft\hfill BA Politikwissenschaft}}


\maketitle[0]

\newpage
\thispagestyle{empty}
\tableofcontents

\newpage
\setcounter{page}{1}

\section{Einleitung}



\section{Theorie}

Weiblichkeit verstehen wir im Folgenden als Diskurs, der als Identitätsangebot dient. In Anlehnung an Lacan kann sich das Subjekt nur über ein von sich selbst verschiedenes Außen konstituieren. "`Durch diese Referenz auf ein Außen wird aber auch eine endgültige Schließung und damit die Etablierung einer \textit{vollen/vollstänidgen} Identität verunmöglicht"' (\cite[199; Hervorh. im Orig.]{nonhoff_kollektive_2007}). Dadurch entsteht ein Mangelempfinden, dass das Subjekt zu schließen versucht, indem es sich mit immer neuen Subjektpositionen identifiziert. Als Identifikationsangebot dienen Diskurse - in unserem Fall über Weiblichkeit - die abgeschlossen erscheinen und ein klare Vorstellung darüber liefern, welche Eigenschaften zu Weiblichkeit gehören und welche nicht. Doch auch Diskurse sind nicht dauerhaft, sondern nur \textit{temporär} fixiert. Diese Fixierung entsteht durch einen Prozess, den Laclau und Mouffe \textit{Artikulation} nennen und bei dem Bedeutung hergestellt wird, indem Bedeutungselemente in Relation zueinander gesetzt werden. Sie können entweder voneinander unterschiedlich sein (\textit{Differenz}) oder einander ähnlich (\textit{Äquivalenz}), wobei Ähnlichkeit sich ausschließlich durch eine gemeinsame Differenz zu einem Bezugselement konstituiert. Bedeutung entsteht also durch ein Netz von Äquivalenzen und Differenzen. Wichtig ist, dass diese Relationen kontingent sind und durch Artikulationen geschaffen wurden. Es können auch immer wieder Bedeutungselemente auftauchen, die die bisherige Einteilung in Äquivalenz und Differenz in Frage stellen und damit den Diskurs verschieben. Deswegen können auch Diskurse niemals vollständig geschlossen sein. Aus einem Mangelempfinden heraus identifiziert sich das Subjekt also mit geschlossen erscheinenden Diskursen, die sich aber ständig verändern, womit der Prozess der Identifikation nie abgeschlossen werden kann.

%steht Kontingenz mit Hegemonie eigentlich im Widerspruch? (siehe weiter unten) Sind sie auch kontingent, wenn sie sich durch "gesellschaftlich Kämpfe ausgebildet haben?"

Wenn auf Subjekte als "`Frau"' oder "`weiblich"' Bezug genommen wird, können sie sich damit identifizieren. In dieser Arbeit soll allerdings nicht untersucht werden, ob dieses Angebot wahrgenommen wird (dies erscheint uns ob der (gewaltvollen) Zweiteilung der Gesellschaft in Männer und Frauen offensichtlich), sondern inwieweit sich der Diskurs \textit{Weiblichkeit} verändert hat. 

Dazu erscheint die Diskurstheorie\footnote{Auf die Theorie wird je nach Schwerpunktsetzung als Hegemonie-, Diskurs oder radikale Demokratie Theorie bezug genommen. Da der Schwerpunkt dieser Arbeit auf der Untersuchung eines Diskurses liegt, wird im Folgenden von einer Diskurstheorie gesprochen (\cite[vgl.][]{nonhoff_diskurs_2007-1}).} von Laclau und Mouffe besonders geeignet, da sie einen differenzierten Diskursbegriff aufweist, der sich nicht nur aus psychoanalytischen Erkenntnissen speist und damit das "`Subjekt in seiner fragmentierten und zerrütteten Verfasstheit beschreiben"' (\cite[198]{nonhoff_kollektive_2007}) kann, sondern auch aus der Hegemonietheorie Gramscis, wodurch sich Diskurse als gesellschaftliche Kämpfe um Hegemonien darstellen lassen, deren temporäre Fixierungen stets angefochten werden können (\cite[vgl.][8-9]{nonhoff_diskurs_2007-1}). Durch die Ausdifferenzierung eines Diskurses in mehrere Bestandteile lassen sich Veränderungen systematisch nachvollziehen.

Ein Diskurs ist bei Laclau und Mouffe ein hegemonial gewordenes, temporär stabilisiertes Bedeutungssystem, das sich aus mehreren Bestandteilen zusammensetzt: \textit{Momenten, Elementen, Knotenpunkten} und \textit{Antagonismen}. Dabei ist ein \textit{Moment} eine artikulierte Position, die im Diskurs schon relative Stabilität erreicht hat, was sich daran zeigt, dass sie häufig wiederholt wird. Ein \textit{Knotenpunkt} ist ein besonders stabiles Moment, das den Diskurs strukturiert, indem es einen Bezugspunkt für alle Momente im Diskurs darstellt. Ein \textit{Antagonismus} wird benötigt, um die Verbindungen der Momente zum Knotenpunkt verstehbar zu machen; allen Momenten eines Knotenpunktes ist gemeinsam, dass sie sich negativ auf den Antagonismus beziehen (\textit{Differenz}). Er ist die "`negative Wendung des Knotenpunktes"' \cite[6]{bruell_chancen_2006}. Wenn ein Knotenpunkt zum Bezugspunkt für besonders viele Momente wird, kann er zum \textit{leeren Signifikanten} werden. Er verliert dabei an inhaltlicher Genauigkeit, bietet aber Raum für verschiedene Bedeutungen und gewinnt so an Macht, da sich zunehmend mehr Subjektpositionen mit ihm identifizieren können. Ein \textit{Element} schließlich ist eine noch nicht intelligible Position in einem Diskurs, die daher unsichtbar ist und immer nur im Rückblick ausgemacht werden kann. Wenn ein neues Moment im Diskurs auftaucht, ist dies ein Hinweis darauf, dass die Position vorher als Element schon im Diskurs vorhanden war und nun hegemonial werden konnte. Ein Diskurs ist von einem Feld "`verstreuter Identitäten"' (\cite[vgl.][6]{bruell_chancen_2006}) umgeben, also einem Feld, in dem Bedeutungen nicht durch einen Diskurs strukturiert sind, das \textit{Diskursivität} genannt wird. 
%leerer Signifikant als Unmöglichkeit? Repräsentiert eine Totalität, die tatsächlich unmöglich ist (siehe laclau 2009 30)

\begin{singlespace*}
\begin{quote}
"`Die soziale Wirklichkeit kann insofern als wesentlich diskursiv verstanden werden, als sie eine sinnhafte Wirklichkeit ist, in der sich die Bedeutung aller sinntragenden Einheiten erst in Relation und damit in Differenz zu anderen Einheiten etabliert. Somit sind Diskurse explizit \textit{nicht} auf die Sphäre der Sprache begrenzt: Auch Objekte, Subjekte, Zustände oder Praktiken ergeben erst im sozialen Relationsgefüge einen je spezifischen Sinn und sind insofern diskursiv strukturiert"' (\cite[9; Hervorh. im Orig.]{nonhoff_diskurs_2007-1}).
\end{quote}
\end{singlespace*}

\noindent Diskurse machen also Handlungen im sozialen Kontext überhaupt erst verstehbar und prägen damit maßgeblich unsere Wahrnehmung der Welt und uns selbst. Sie sind laut Ernesto Laclau "'In diesem Sinne [...] gleichbedeutend mit dem gesellschaftlichen Leben" (\cite[29]{nonhoff_ideologie_2007}). Für die vorliegende Arbeit bedeutet das, dass wir Geschlecht nicht als natürlich gegeben verstehen, sondern als von Diskursen strukturiert und gesellschaftlich konstruiert, folglich aber eben nicht dauerhaft fixiert, sondern wandelbar. Im Folgenden soll untersucht werden, inwieweit sich der Diskurs um Weiblichkeit, spezifischer die Anforderungen an Frauen im Nationalsozialismus mit Beginn des Krieges verändert haben. Dazu sollen die Momente des Knotenpunktes "`Frau sein"' des Diskurses "`Weiblichkeit"' vor und nach Kriegsausbruch ausgemacht werden. Um die These stützen zu können, dass sich mit Kriegsbeginn der Diskurs um Weiblichkeit verschoben hat, müssen bei der Analyse des Materials vor Ausbruch des Krieges andere Momente ausgemacht werden können, als nach Ausbruch des Krieges. 

\section{Operationalisierung und Methode}

Nach der Diskurstheorie von Laclau und Mouffe, funktionieren Identität und Identifikation nur vermittelt über Diskurse, verstanden als temporär fixierte Bedeutungssysteme. Eine Verschiebung der Weiblichkeitsvorstellungen lässt sich folglich durch eine Analyse des zugrundeliegenden Diskurses nachweisen. Konkret bedeutet dies, dass sich in der Diskursstruktur nach 1939 neue Momente finden müssen, welche an den Knotenpunkt "`Frau sein"' gebunden werden und somit eine andere Identifikation ermöglichen. Zeitgleich müssen diese qualitativ stark genug sein, um nicht sofort wieder aus dem Diskurs zu verschwinden. Die Frage ist nun, wie die Diskursstruktur analysiert werden kann? Die von Cornelia Bruell und Monika Mokre vorgeschlagene \textit{Simultanzanalyse} leistet genau dies (\cite*{bruell_chancen_2006}; siehe auch \cite{nonhoff_kollektive_2007}). Im folgenden wird diese Operationalisierung von Laclau und Mouffes Diskurstheorie vorgestellt um anschließend kurz auf die dazu notwendige Methode der \textit{qualitativen Inhaltsanalyse} einzugehen.

\subsection{Die \textit{Simultanzanalyse} nach Bruell/Mokre}

Diskurse stellen sich als ein komplexes, sich ständig wandelndes Feld aus sozialen Handlungen dar. Um sie überhaupt empirisch fassen zu können, muss also ein Einschnitt gemacht werden. Medien sind hierfür gut geeignet, da sie "`eine Momentaufnahme eines Ausschnittes komplex vernetzter Diskurse"' (\cite[8]{bruell_chancen_2006}) ermöglichen. Sie sind ein Ort der Konstitution von Diskursen, stellen jedoch immer nur einen Ausschnitt dar, weshalb die Wahl des Mediums wohlüberlegt und begründet sein sollte. Zeitgleich sind Medien "`für die Analyse von kollektiven Identifikationsmustern besonders geeignet, da die hegemoniale Konstruktion von Bedeutungsnetzwerken vor allem dann erfolgreich ist, wenn sie sich in öffentlichen Debatten durch Wiederholungen und durch ein ausgedehntes Netzwerk von Anknüpfungspunkten manifestiert"' (\cite[202]{nonhoff_kollektive_2007}). Weitreichende kollektive Identifikationen -- wie der Diskurs um Weiblichkeit -- entstehen immer hauptsächlich in der Öffentlichkeit. 

Zur Analyse der Diskursstruktur, müssen die oben beschriebenen Bestandteile des Diskurses mithilfe einer \textit{Simultanzanalyse} identifiziert werden. Simultanzen sind dabei die Form des gemeinsamen Vorkommens von Momenten, also deren Überlappen oder Aufeinandertreffen (\cite[11][vgl.]{bruell_chancen_2006}). Dabei ist es nach der Diskurstheorie von Laclau/Mouffe irrelevant, ob die aussagen argumentativ verknüpft sind oder nicht; die Einheit besteht durch den Text. Ob eine interpretative Verknüpfung tatsächlich durch die einzelne Leser\_in geschieht, ist für die Diskursstruktur nicht wichtig. "`Es können also auch lose und nicht inhaltlich aneinandergereihte Aussagen über die Rezeption zur Diskursstruktur beitragen, obwohl natürlich argumentativ verknüpfte Einheiten eine vernetzende Interpretation forcieren"' (\cite[9]{bruell_chancen_2006}). Zur Erhebung der Simultanzen werden zunächst Aussagen und Positionen, mithilfe einer qualitativen Inhaltsanalyse, zu Kategorien zusammengefasst. Anschließend wird eine qualitativ starke Kategorie als Ausgangspunkt genommen und erfasst, welche anderen Kategorien mit dieser überlappen (\cite[vgl.][205]{nonhoff_kollektive_2007}). Dies wird mit allen qualitativ relevanten Kategorien wiederholt.

Simultanzen können nun qualitativ stark oder schwach sein, je nach der absoluten Häufigkeit der Vergleichskategorie. Außerdem können sie qualitativ stark oder schwach sein, je nach der relativen Häufigkeit des gleichzeitigen Vorkommens der Kategorien. Überlappt also eine Vergleichskategorie beispielsweise in 9 von 10 Fällen mit der Ausgangskategorie, so kann von einer qualitativ starken Simultanz gesprochen werden. Daraus ergeben sich folgende vier Typen von Simultanzen (\cite[vgl.][12]{bruell_chancen_2006}:

\noindent Die \textit{Äquivalenzierende Simultanz} (quantitativ/qualitativ stark) verweist entweder auf eine starke Aquivalenzkette der zwei Momente oder, wenn eine Kategorie mit vielen anderen stark simultan vorkommt, auf einen \textit{Knotenpunkt}.

\noindent Die \textit{lösende Simultanz} (quantitativ stark/qualitativ schwach) lässt Vermuten, dass die Vergleichskategorie Teil eines anderen Diskurses ist, der mit dem untersuchten Diskurs überlappt.

\noindent Bei der \textit{unterordnenden Simultanz} (quantitativ schwach/qualitativ stark) kann davon ausgegangen werden, dass die Vergleichskategorie einen Moment der Ausgangskategorie bildet. Hat eine Ausgangskategorie mehrere dieser Simultanzen, bildet diese mit Sicherheit einen Knotenpunkt.

\noindent Die \textit{differenzierende Simultanz} (quantitativ/qualitativ schwach) verweist auf Momente, die entweder hauptsächlich in anderen Diskursen gebunden werden oder sich im Feld der \textit{Diskursivität} befinden.

\subsection{Qualitative Inhaltsanalyse}

Philipp Mayring benennt als Spezifika der qualitativen Inhaltsanalyse eine systematische, regel- und theoriegeleitete Analyse einer fixierten Kommunikation, mit dem Ziel Rückschlüsse auf bestimmte Aspekte eben dieser Kommunikation zu ziehen (vgl. \cite*[13]{mayring_qualitative_2010}). Wichtig ist in erster Linie der Vorrang, den inhaltliche Argumente gegenüber Verfahrensargumenten einnehmen; die Verfahren sollen sich immer am Material orientieren und müssen daher stets für die jeweilige Forschungsfrage angepasst werden. Wie die jeweilige Technik konkret ausgestaltet wird, muss theoriegeleitet begründet werden, was vor allem für die Explikation der Fragestellung gilt (vgl. \cite[50-51]{mayring_qualitative_2010}). Mayring betont, dass die Inhaltsanalyse keine "`freie"' Interpretation ist, sondern jeder Analyseschritt durch im Vorhinein festgelegte Regeln intersubjektiv nachvollziehbar sein muss. Im Zentrum der Analyse steht das \textit{Kategoriensystem}, dessen Konstruktion und Anwendung interpretativ erfolgt und das eine Vergleichbarkeit der Ergebnisse überhaupt erst ermöglicht (vgl. \cite[49]{mayring_qualitative_2010}). Margrit Schreier sieht in der Fokussierung auf Kategorien das zentrale Differenzierungsmerkmal gegenüber anderen qualitativen Methoden (vgl. \cite[3]{schreier_varianten_2014}).

In dieser Arbeit wird die Variante der inhaltlich-strukturierenden Inhaltsanalyse nach Mayring mit einigen methodischen Ergänzungen von Margrit Schreier angewendet, wobei das Kategoriensystem vollständig induktiv mit der Technik der zusammenfassenden Inhaltsanalyse nach Mayring erstellt wird (vgl. \cite{mayring_qualitative_2010}; \cite{schreier_varianten_2014}). Beide Wissenschaftler\_innen sehen die strukturierende Variante der Inhaltsanalyse als die zentrale und definieren diese nahezu identisch:

\begin{singlespace*}
\begin{quote}
"`Kern der inhaltlich-strukturierenden Vorgehensweise ist es, am Material ausgewählte inhaltliche Aspekte zu identifizieren, zu konzeptualisieren und das Material im Hinblick auf solche Aspekte systematisch zu beschreiben. [...] Diese Aspekte bilden zugleich die Struktur des Kategoriensystems; die verschiedenen Themen werden als Kategorien des Kategoriensystems expliziert"' (\cite[5]{schreier_varianten_2014}).
\end{quote}
\end{singlespace*}

\noindent Es gibt nun unterschiedliche Möglichkeiten das Kategoriensystem zu erstellen. Üblicherweise werden deduktiv Oberkategorien gebildet und anschließend induktiv Unterkategorien durch Subsumtion diesen Oberkategorien zugeordnet.\footnote{Mayring lässt an dieser Stelle keine induktive Kategorienbildung zu, vgl. kritisch dazu \textcite[Kap. II.4]{steigleder_strukturierende_2008}.} Für die \textit{Simultanzanalyse} macht es jedoch keinen Sinn Kategorien schon im Vorhinein festzulegen, sie müssen notwendigerweise aus dem Material entwickelt werden. 

Als Verfahren zur Kategorienbildung wird die zusammenfassende Inhaltsanalyse verwendet (vgl. \cite[Kap. 5.5.2]{mayring_qualitative_2010}). Mayring sieht die Zusammenfassung als eigene Grundform der qualitativen Inhaltsanalyse, was in der Literatur allerdings umstritten ist. Schreier sieht diese eher als Möglichkeit "`für die Generierung inhaltlich-thematischer Kategorien im Rahmen eines qualitativ-strukturierten inhaltsanalytischen Vorgehens"' (\cite[14]{schreier_varianten_2014}).\footnote{Eine Tabelle mit den Zusammenfassungen, sowie die Codings werden vorgehalten und können bei den Autor\_innen nachgefragt werden.} Die anschließende systematische Beschreibung des Materials erfolgt auf Grundlage der oben beschriebenen \textit{Simultanzanalyse}.

Um die Nachvollziehbarkeit der Analyse zu gewährleisten, folgt die inhaltlich-strukturierende Inhaltsanalyse einem vorher festgelegten Ablaufmodell. Zunächst wird das Ausgangsmaterial bestimmt und in ein Kommunikationsmodell eingeordnet, wobei vor allem die Festlegung, Entstehungssituation und formale Charakteristika erläutert werden (vgl. \cite[52-53]{mayring_qualitative_2010}). Anschließend wird kurz die Richtung der Analyse anhand des Materials erläutert, um dann mit dem Verfahren der zusammenfassenden Inhaltsanalyse das Kategoriensystem zu erstellen, welches in einem letzten Schritt systematisch, unter Rückbezug auf die Theorie beschrieben wird.

\subsection{Vorgehen}

%Material
Als Analysematerial wurden sieben Zeitschriftenartikel aus der Propagandazeitschrift \textit{NS-Frauen-Warte} ausgewählt, wobei drei Artikel aus dem Jahrgang 1935/36 und vier aus dem Jahrgang 1941 stammen. Die \textit{NS-Frauen-Warte} war die Zeitschrift der NS-Frauenschaft (NSF) -- der Frauenorganisation innerhalb der NSDAP -- richtete sich jedoch an ein größeres Publikum. Die Auflage lag 1936 bei ca. einer halben Million und stieg dann bis 1939 auf 1,5 Millionen an (\cite[vgl.][89-90]{dohring_von_2004}). Wir haben die Zeitschrift aus zwei Gründen für die Analyse ausgesucht: Einerseits erreichte sie mit ihrer Auflage relativ viele Frauen, andererseits ist sie das Zentrale Organ der NS-Führung von und für Frauen. Sollte eine Hegemonieverschiebung in der Identifikation mit "`Frau sein"' nach 1939 stattgefunden haben, so liegt die Vermutung nahe, dass diese von der Führungselite ausging und sich daher auch in deren Publikationen niederschlägt. Die Stichprobe haben wir, nach einer Durchsicht der Jahrgänge, vor allem nach ihrer Anschaulichkeit ausgewählt, wobei wir versucht haben, die absolute Anzahl der Findings für beide Jahrgänge in etwa gleich groß zu halten. Die Auswahl kann also nicht als repräsentativ gelten, sondern stellt einen Pre-Test dar.

%Analyserichtung
Die Richtung der Analyse beginnt bei einzelnen Aussagen in den Texten und endet mit der Identifikation der Diskursstrukturen. Im einzelnen läuft die Analyse also folgendermaßen ab: zunächst wird das Kategoriensystem entwickelt, anschließend wird die Häufigkeit der einzelnen Kategorien festgestellt. Dann werden die Simultanzen analysiert, um so die Diskursbestandteile zu identifizieren. Abschließend werden die Diskursstrukturen der Jahrgänge verglichen um die Fragestellung zu beantworten.

Das Ablaufmodell zur Entwicklung des Kategoriensystems stellt sich wie folgt dar: Zunächst werden die Analyseeinheiten, also Kodier-, Kontext- und Auswertungseinheit festgelegt. Anschließend werden die inhaltstragenden Textstellen markiert und in Kategorien zusammengefasst, wobei jede neue Textstelle, wenn diese schon durch eine Kategorie abgedeckt ist, einer schon vorhandenen Kategorie zugeordnet wird. Ist dies nicht der Fall, wird eine neue Kategorie erstellt. So wird das Material verdichtet und kann über mehrere Aussagen hinweg zusammengefasst werden. Schreier nennt dieses Verfahren zur induktiven Kategorienbildung "`Subsumtion"' (\cite[7]{schreier_varianten_2014}).

Bei der zusammenfassenden Inhaltsanalyse fallen Kontext- und Auswertungseinheit zusammen (vgl. \cite[71]{mayring_qualitative_2010}), sie definieren, welches der größte Textbestandteil ist, der unter eine Kategorie fallen kann (vgl. \cite[59]{mayring_qualitative_2010}). Da sich die Simultanzanalyse auf Überlappungen der Kategorien innerhalb von Aussagen bezieht, muss diese Einheit auf den einzelnen Artikel festgelegt werden. Die Kodiereinheit "`legt fest, [...] was der minimale Textteil ist, der unter eine Kategorie fallen kann"' (\cite[59]{mayring_qualitative_2010}) und ist in diesem Fall jede vollständige Aussage über Vorstellungen, Verhaltensweisen und Charakterzüge in Bezug auf "`Frau sein"'.

\section{Auswertung}

Bei der qualitativen Inhaltsanalyse wurden den einzelnen Aussagen über Vorstellungen und Verhalten von Frauen induktiv Kategorien zugeordnet, die die Aussagen zusammenfassen und sich systematisieren lassen. Einer Aussage konnten auch mehrere Kategorien zugeordnet werden. Folgende Kategorien wurden dabei gefunden: \textit{Vorbildfunktion, Pflichtbewusstsein gegenüber der Nation, schweigsam, nüchtern/sachlich, mutig/tapfer, Organisation/Selbständigkeit, Ehe, selbstlos/bescheiden, Sorgearbeit/hilfsbereit, Frauenarbeitsdienst, Gläubigkeit, Mutter, instinktiv, Häuslichkeit/Hausarbeit, "`Natur der Frau"', Berufstätigkeit, Empathie} und \textit{friedliebend}. Die Kategorien \textit{schweigsam, nüchtern/sachlich, Frauenarbeitsdienst} und \textit{Gläubigkeit} wurden jeweils nur ein oder zweimal gefunden und können deswegen im Folgenden vernachlässigt werden. Die Kategorien \textit{Sorgearbeit, Mutter, Berufstätigkeit, Ehe} und \textit{Pflichtbewusstsein} hingegen treten sehr häufig auf (insgesamt jeweils 17 bis 32 mal). Wenn ein Moment eine Artikulation ist, die sich durch ihre häufige Wiederholung auszeichnet, dann kann davon ausgegangen werden, dass diese fünf Kategorien Momente des Knotenpunktes "`Frau sein"' bilden. Auch die Kategorie \textit{mutig/tapfer} tritt häufig auf (22 mal), doch bei näherer Betrachtung ist dies der überproportional häufigen Erwähnung in einem einzigen Artikel geschuldet (13 mal in \textit{Die deutsche Frau im Kriege}). Sie wird daher nicht als Moment behandelt. 
 
Die Kategorien \textit{Sorgearbeit, Mutter, Berufstätigkeit, Ehe} und \textit{Pflichtbewusstsein} zeigen sowohl in den Zeitungsartikeln von 1936 wie auch in denen von 1941 häufiges Vorkommen und können als konstante Momente gelten. Die Kategorie \textit{Berufstätigkeit} tritt erst 1941 stark auf. Sie war schon 1936 vorhanden, konnte aber mit Kriegsausbruch hegemonial werden und sich zu einem Moment etablieren. Dies lässt dennoch nicht -- wie angenommen -- darauf schließen, dass sich der Diskurs um Weiblichkeit mit Ausbruch des Krieges grundlegend verändert hat. Dazu hätten sich die gefundenen Momente 1936 und 1941 stärker voneinander unterscheiden müssen, hier bleiben sie jedoch konstant und der Knotenpunkt wird bei Kriegsausbruch lediglich um ein neues Moment erweitert.

Um die Struktur des Diskurses genauer untersuchen zu können und einen möglichen Einfluss des neu hinzugekommenen Moments zu erkennen, wurde im Anschluss eine Simultanzanalyse mit den festgestellten Momenten für die Jahre 1936 und 1941 durchgeführt. Da keines von ihnen mehrere andere Kategorien an sich bindet, also keine der Ausgangskategorien starke Simultanzen mit mehreren Vergleichskategorien aufweist, kann sicher gesagt werden, dass es sich bei allen Momenten tatsächlich um solche handelt und nicht etwa um weitere Knotenpunkte im Diskurs. Die Momente \textit{Ehe} und \textit{Mutter} treten nach Kriegsbeginn häufiger simultan in Erscheinung als vorher und binden außerdem die Kategorie \textit{Häuslichkeit} stärker an sich. \textit{Berufstätigkeit} hingegen zeigt kaum Simultanzen mit \textit{Ehe} und \textit{Mutter}. Dies lässt darauf schließen, dass \textit{Ehe} und \textit{Mutter} hier eine Äquivalenzkette bilden und sich nach Kriegsausbruch zwei antagonisierende Identifikationsangebote für Frauen entwickelten: Zum einen das der heimischen, verheirateten Mutter, zum anderen das der jungen\footnote{Tatsächlich tritt die Kategorie \textit{Berufstätigkeit} besonders häufig auf, wenn von jungen Frauen die Rede ist. Dies wurde allerdings nicht mitcodiert, da \textit{jung} als Identifikationsangebot für Frauen im Allgemeinen wenig sinnvoll erschien.}, unverheirateten berufstätigen Frau. 

Zu beobachten ist außerdem, dass \textit{Berufstätigkeit} und \textit{Sorgearbeit} einen schwachen Zusammenhang aufweisen. Es ist zu vermuten, dass Berufstätigkeit oft mit Sorgearbeit verknüpft wurde, Frauen zwar nicht zwingend soziale Berufe ergreifen mussten, dies jedoch häufig geschah und gern gesehen war. Verwunderlich scheint, dass \textit{Sorgearbeit} kaum Simultanzen mit \textit{Ehe} und \textit{Mutter} aufweist, aber \textit{Organisation/Selbständigkeit} an sich bindet. 
%muss ich noch eine Erklärung für finden

\textit{Pflichtbewusstsein gegenüber der Nation} bindet keine anderen Kategorien an sich und zeigt auch kaum Simultanzen. Es scheint als \textit{lösende Simultanz} als Knotenpunkt einen anderen Diskurs zu strukturieren und für die Identifikation als Frau nicht ausschlaggebend, aber auch nicht irrelevant zu sein. Dies ergibt im Kontext des Nationalsozialismus durchaus Sinn, da Nationalismus und damit Pflichtbewusstsein gegenüber Deutschland den wohl wichtigsten Bestandteil der Ideologie ausmacht und wesentlich zur Identifikation als Deutsche\_r nicht nur für Frauen beiträgt.


Da sich hier unter dem gleichen Knotenpunkt zwei antagonisierende Äquivalenzketten bilden und weitere Momente gebunden werden, die zu diesen Äquivalenzketten keine direkte Verbindung zeigen, kann darauf geschlossen werden, dass es sich bei "`Frau sein"' nicht nur um einen gewöhnlichen Knotenpunkt handelt, sondern um einen \textit{leeren Signifikanten}. Dies kommt vor, wenn ein Knotenpunkt mit besonders vielen Bedeutungen gefüllt wird, wodurch er zwar an Inhalt verliert, aber auch machtvoller wird, da er mehr Momente an sich binden kann und somit für mehr Subjekte Anhaltspunkt zur Identifikation bietet. Dies deckt sich mit Alltagserfahrungen, wo "`Frau"' als Identität für die unterschiedlichsten Subjekte, mit unterschiedlichsten Eigenschaften, Erfahrungen und Meinungen fungiert.
%macht das Sinn im Verlgeich zu dem, was wir gerade erforschen? 

%auf kleines n eingehen

\section{Fazit}



\newpage

\nocite{medici_faschistische_1941}
\nocite{a._v._s._kameradschaft_1936}
\nocite{maltzahn_deutsche_1936}
\nocite{reimer_glucklich_1941}
\nocite{scholtz-klink_frauen_1936}
\nocite{weinhandl_wie_1941}
\nocite{friewart_japanisches_1941}


\printbibheading[title=Literaturverzeichnis]
\printbibliography[heading=subbibliography, keyword={Quelle}, title={Quellen}]
\printbibliography[heading=subbibliography, notkeyword={Quelle}]
\newpage

\appendix

\addpart{\appendixname}

\section{Kodierleitfaden}

\subsection*{Kategoriensystem}

{\raggedright

\vspace{3pt} \noindent
\begin{tabular}{p{145pt}p{10pt}}
\parbox{145pt}{\raggedright 
{\small \textbf{1 (Un-)Weiblichkeit}}
} \\
\hline
\parbox{145pt}{\raggedright 
{\small      1.1 Aussehen}
} & \parbox{10pt}{\raggedright 
{\small 3}
} \\
\hline
\parbox{145pt}{\raggedright 
{\small      1.2 Sexuelle Aktivität}
} & \parbox{10pt}{\raggedright 
{\small 3}
} \\
\hline
\parbox{145pt}{\raggedright 
{\small      1.3 Schwangerschaft}
} & \parbox{10pt}{\raggedright 
{\small 4}
} \\
\hline
\parbox{150pt}{\raggedright 
{\small \textbf{2 Absprechen von Rationalität}}
} \\
\hline
\parbox{145pt}{\raggedright 
{\small      2.1 Geringes Alter}
} & \parbox{10pt}{\raggedright 
{\small 6}
} \\
\hline
\parbox{145pt}{\raggedright 
{\small      2.2 Geistige Beeinträchtigung}
} & \parbox{10pt}{\raggedright 
{\small 9}
} \\
\hline
\parbox{145pt}{\raggedright 
{\small      2.3 Soziale Herkunft}
} & \parbox{10pt}{\raggedright 
{\small 4}
} \\
\hline
\parbox{145pt}{\raggedright 
{\small \textbf{3 Emotionen}}
} \\
\hline
\parbox{145pt}{\raggedright 
{\small      3.1 Scherz}
} & \parbox{10pt}{\raggedright 
{\small 3}
} \\
\hline
\parbox{145pt}{\raggedright 
{\small      3.2 Zeigt Ausdruck von Freude}
} & \parbox{10pt}{\raggedright 
{\small 2}
} \\
\hline
\parbox{145pt}{\raggedright 
{\small      3.3 Genuss}
} & \parbox{10pt}{\raggedright 
{\small 4}
} \\
\hline
\parbox{145pt}{\raggedright 
{\small      3.4 Sadismus}
} & \parbox{10pt}{\raggedright 
{\small 9}
} \\
\hline
\parbox{145pt}{\raggedright 
{\small \textbf{4 Dämonisierung}}
} & \parbox{10pt}{\raggedright 
{\small 2}
} \\
\hline
\parbox{145pt}{\raggedright 
{\small \textbf{5 Zeigen auf Genitalien}}
} & \parbox{10pt}{\raggedright 
{\small 4}
} \\
\hline
\parbox{145pt}{\raggedright 
{\small \textbf{6 Leine}}
} & \parbox{10pt}{\raggedright 
{\small 7}
} \\
\hline
\parbox{145pt}{\raggedright 
{\small \textbf{7 Verhältnis zu/mit Graner}}
} & \parbox{10pt}{\raggedright 
{\small 12}
} \\
\hline
\parbox{145pt}{\raggedright 
{\small \textbf{8 Einfluss von Anderen}}
} \\
\hline
\parbox{145pt}{\raggedright 
{\small      8.1 Ist überredet worden}
} & \parbox{10pt}{\raggedright 
{\small 7}
} \\
\hline
\parbox{145pt}{\raggedright 
{\small      8.2 Hat Befehle ausgeführt}
} & \parbox{10pt}{\raggedright 
{\small 2}
} \\
\hline
\end{tabular}
\vspace{2pt}

}

\newpage

\subsection*{Vorbildfunktion}

Kodiert werden alle Aussagen, die Tätigkeiten von Frauen als beispielhaft hervorheben oder zu ihrer Nachahmung animieren.\\
Ankerbeispiel "`Und der Einsatz der Frauen fern der Heimat ist, wo immer wir auch von ihm hören, beispielhaft"' (\cite[669]{maltzahn_deutsche_1936}).

\subsection*{Pflichtbewusstsein gegenüber der Nation}

Kodiert werden alle Aussagen, die Handlungen von Frauen in einen Kontext von Nationalismus und Ehrgefühl gegenüber einer "`größeren Aufgabe"' stellen.\\
Ankerbeispiel: "'Ihr Wirken ist selbstverständliches Einfügen in die große Aufgabe gewesen, so daß es seinen schweren und tapferen Teil zur Unbesiegbarkeit in Ostafrika beigetragen hat"' (\cite[669]{maltzahn_deutsche_1936}).

\subsection*{schweigsam}

Kodiert werden alle Aussagen, die auf Schweigsamkeit von Frauen hinweisen.\\
Ankerbeispiel: "`Noch viel zu wenig wissen wir von ihnen, die mit Heimweh und der brennenden Sorge um Deutschland im Herzen ihr Pflicht taten, sehr selbstverständlich und sehr schweigsam"' (\cite[669]{maltzahn_deutsche_1936}).

\subsection*{nüchtern, sachlich}

Kodiert werden alle Aussagen, die auf eine sachliche Ausdrucksweise von Frauen hinweisen.\\
Ankerbeispiel: "'Es ist so nüchtern und sachlich geschrieben wie nur möglich. Jede Ausschmückung, jede Stimmung fehlt."' (\cite[669]{maltzahn_deutsche_1936}).

\subsection*{mutig, tapfer}

Kodiert werden alle Aussagen, die Handlungen von Frauen als tapfer darstellen, indem sie meist in einen Kontext unangenehmer Umstände gesetzt werden.\\
Ankerbeispiel: "`Die 80 Frauen, die das austeilen, haben wohl keinen trockenen Faden am Leib und stundenlang gestanden, aber Scherzworte fliegen hinüber und herüber, und die Augen strahlen"' (\cite[775]{a._v._s._kameradschaft_1936}).

\subsection*{Organisation, Selbständigkeit}

Kodiert werden alle Aussagen, die beschreiben, dass Frauen federführend die Leitung einer Organisation innehaben oder selbständig und/oder in Eigeninitiative Tätigkeiten organisieren.\\
Ankerbeispiel: "`[...] und es wird wahrhaftig fieberhaft gearbeitet, damit die weitverzweigte und wohldurchdachte Organisation, deren Fäden alle in den Händen der Gaufrauenschaftsleiterin zusammenlaufen, wie am Schnürchen funktioniert"' (\cite[778]{a._v._s._kameradschaft_1936}).

\subsection*{Ehe}

Kodiert werden alle Aussagen, die Frauen in Verbindung zu ihrem Ehemann setzen.\\
Ankerbeispiel: "`Denn die Handwerkerfrau ist wieder wie die Bauersfrau die geeignetste Frau, um den jungen deutschen Nachwuchs zu erziehen"' (\cite[836]{scholtz-klink_frauen_1936}).

\subsection*{selbstlos, bescheiden}

Kodiert werden alle Aussagen, die darauf Hinweisen, dass Frauen Tätigkeiten ausführen, ohne sich dabei um ihre eigenen Bedürfnisse zu kümmern.\\
Ankerbeispiel: "`Das mag nicht immer leicht sein, aber schon als kleines Mädchen hat sie gelernt, darauf zu achten, daß erst alle anderen Familienmitglieder versorgt waren, ehe sie selber ihr Mahl verzehrte"' (\cite[5]{friewart_japanisches_1941}).

\subsection*{Sorgearbeit}

Kodiert werden alle Aussagen, die beschreiben, dass sich Frauen um andere kümmern und sie versorgen oder Sorge um andere Menschen äußern.\\
Ankerbeispiel: "`Auf jedem Bahnhof sind faschistische Frauen Tag und Nacht tätig, um die durchreisenden Mannschaften mit Lebensmitteln zu versorgen"' (\cite[3]{medici_faschistische_1941}).

\subsection*{Frauenarbeitsdienst}

Kodiert werden alle Aussagen, die darauf hinweisen, dass Frauen für den Frauenarbeitsdienst arbeiten oder durch ihn unterstützt werden.\\
Ankerbeispiel: "`Denn im Frauenarbeitsdienst sehen wir die große Schule, durch die einmal unsere ganze neuen Frauengeneration hindurchgehen wird, und von der wir uns den allergrößten Erfolg versprechen"' (\cite[775]{a._v._s._kameradschaft_1936}).

\subsection*{Gläubigkeit}
Kodiert werden alle Aussagen, die auf Religiosität von Frauen hinweisen. \\
Ankerbeispiel: "`O nein, hilf dir selbst und sieh Gottes Hilfe darin, daß er dir Geist und Verstand, Kraft und Ausdauer gab"' (\cite[669]{maltzahn_deutsche_1936}).

\subsection*{Mutter}

Kodiert werden alle Aussagen, die Frauen in Verbindung mit Mutterschaft bringen, in denen sie entweder explizit so betitelt werden, oder ihr Handeln auf ihre Kinder bezogen wird.\\
Ankerbeispiel: "`Was denkt sie wohl, während sie behutsam und geschickt zufaßt? Es ist unschwer zu erraten: sie denkt an das eigene [Kind], das sie einmal haben möchte"' (\cite[775]{a._v._s._kameradschaft_1936}).

\subsection*{instinktiv}

Kodiert werden alle Aussagen, die das Handeln von Frauen als intuitiv bewerten.\\
Ankerbeispiel: "`Immer spürt dann die echt Frau instinktmäßig, was sie sagen, tun, unternehmen muß, um die gestörte Eintracht zu bewahren"' (\cite[36]{weinhandl_wie_1941}).

\subsection*{Häuslichkeit, Hausarbeit}

Kodiert werden alle Aussagen, die Tätigkeit von Frauen im Haus und Haushalt beschreiben.\\
Ankerbeispiel: "`Die Frau des Hauses aber kann sich ungestört der Hauptaufgabe ihres Lebens widmen, dem Manne eine Stätte des ruhigen Behagens, der Erholung und der häuslichen Freuden zu bereiten"' (\cite[5]{friewart_japanisches_1941}).

\subsection*{"`Natur der Frau"'}

Kodiert werden alle Aussagen, die bestimmte Tätigkeiten oder Eigenschaften von Frauen als natürliche Wesensmerkmale bewerten.\\
Ankerbeispiel: \enquote{Nichts wird den Frauen tatsächlich \enquote{beigebracht}, es wird nur aus ihnen herausgeholt und vertieft und durch praktisches Wissen erweitert, was an sich in jeder Frau drinnen liegt} (\cite[775]{a._v._s._kameradschaft_1936}).

\subsection*{Berufstätigkeit}

Kodiert werden alle Aussagen, die berufstätige Frauen beschreiben, oder über Berufe, die von Frauen ergriffen werden, schreiben. Um Berufstätigkeit von ehrenamtlichen Tätigkeiten zu unterscheiden, wurden aus Anforderungen an Berufe gesetzt, dass sie eine Ausbildung benötigen und vergütet werden.\\
Ankerbeispiel: "`Der Frauenhilfsdienst für Wohlfahrts- und Krankenpflege, der vor drei Jahren von der Reichsfrauenführerin ins Leben gerufen wurde, war eine Notmaßnahme, um den großen Mangel an Fachkräften in den sozialen Frauenberufen zu überbrücken"' (\cite[22]{reimer_glucklich_1941}).

\subsection*{Empathie}

Kodiert werden alle Aussagen, die beschreiben, dass sich Frauen empathisch in die Gefühlslagen ihrer Mitmenschen hineinversetzen können oder im Umgang mit Menschen besonderes Feingefühl zeigen.\\
Ankerbeispiel: "`Kaum eine Fähigkeit des Frauengemüts wird so bewundert als die Gabe, das, was andere fühlen, so lebhaft in sich mitzuempfinden, wie ein feingestimmtes Instrument, dessen edles Holz alle auf seinen Saiten gespielten Töne nicht nur widerschwingt, sondern auch zu einem schönen Zusammenklang verschmilzt"' (\cite[36]{weinhandl_wie_1941}).

\subsection*{friedliebend}

Kodiert werden alle Aussagen, die Frauen einen besonderen Sinn für oder ein besonderes Bedürfnis nach Harmonie zuschreiben.\\
Ankerbeispiel: "`Diese Fähigkeit der Frau ist dann um so höher zu schätzen, wenn sie in völlig neuen Lagen das Richtige zu treffen und das Bedrohliche zu wenden vermag, kraft ihres sicheren Vorgefühls für alles, was Keime des Mißverstehens, Befremdens, Entzweiens in sich trägt"' (\cite[36]{weinhandl_wie_1941}).

\end{document}
